%%%%%%%%%%%%%%%%%%%%%%%%%%%%%%%%%%%%%%%%%
%
% CMPT 435
% Lab Zero
%
%%%%%%%%%%%%%%%%%%%%%%%%%%%%%%%%%%%%%%%%%

%%%%%%%%%%%%%%%%%%%%%%%%%%%%%%%%%%%%%%%%%
% Short Sectioned Assignment
% LaTeX Template
% Version 1.0 (5/5/12)
%
% This template has been downloaded from: http://www.LaTeXTemplates.com
% Original author: % Frits Wenneker (http://www.howtotex.com)
% License: CC BY-NC-SA 3.0 (http://creativecommons.org/licenses/by-nc-sa/3.0/)
% Modified by Alan G. Labouseur  - alan@labouseur.com, and Patrick Tyler - Patrick.Tyler1@marist.edu
%
%%%%%%%%%%%%%%%%%%%%%%%%%%%%%%%%%%%%%%%%%

%----------------------------------------------------------------------------------------
%	PACKAGES AND OTHER DOCUMENT CONFIGURATIONS
%----------------------------------------------------------------------------------------

\documentclass[letterpaper, 10pt]{article} 

\usepackage[english]{babel} % English language/hyphenation
\usepackage{graphicx}
\usepackage[lined,linesnumbered,commentsnumbered]{algorithm2e}
\usepackage{listings}
\usepackage{fancyhdr} % Custom headers and footers
\pagestyle{fancyplain} % Makes all pages in the document conform to the custom headers and footers
\usepackage{lastpage}
\usepackage{url}
\usepackage{xcolor}
\usepackage{titlesec}
\usepackage{xurl}
\usepackage{multicol}

% Stolen from https://www.overleaf.com/learn/latex/Code_listing 
\definecolor{codegreen}{rgb}{0,0.6,0}
\definecolor{codegray}{rgb}{0.5,0.5,0.5}
\definecolor{codepurple}{rgb}{0.58,0,0.82}
\definecolor{backcolour}{rgb}{0.95,0.95,0.92}

\lstdefinestyle{mystyle}{
    backgroundcolor=\color{backcolour},   
    commentstyle=\color{codegreen},
    keywordstyle=\color{magenta},
    numberstyle=\tiny\color{codegray},
    stringstyle=\color{codepurple},
    basicstyle=\ttfamily\footnotesize,
    breakatwhitespace=false,         
    breaklines=true,                 
    captionpos=b,                    
    keepspaces=true,                 
    numbers=left,                    
    numbersep=5pt,                  
    showspaces=false,                
    showstringspaces=false,
    showtabs=false,                  
    tabsize=2
}
\lstset{style=mystyle, language=c++}


\fancyhead{} % No page header - if you want one, create it in the same way as the footers below
\fancyfoot[L]{} % Empty left footer
\fancyfoot[R]{}

\renewcommand{\headrulewidth}{0pt} % Remove header underlines
\renewcommand{\footrulewidth}{0pt} % Remove footer underlines
\setlength{\headheight}{13.6pt} % Customize the height of the header

%----------------------------------------------------------------------------------------
%	TITLE SECTION
%----------------------------------------------------------------------------------------

\newcommand{\horrule}[1]{\rule{\linewidth}{#1}} % Create horizontal rule command with 1 argument of height

\title{	
   \normalfont \normalsize 
   \textsc{CMPT 308 - Spring 2024 - Dr. Labouseur} \\[10pt] % Header stuff.
   \horrule{0.5pt} \\[0.25cm] 	% Top horizontal rule
   \huge Lab 1 -- ~ Lexer\\     	    % Assignment title
   \horrule{0.5pt} \\[0.25cm] 	% Bottom horizontal rule
}

\author{Patrick Tyler \\ \normalsize Patrick.Tyler1@marist.edu}

\date{\normalsize\today} 	% Today's date.

\begin{document}

\maketitle % Print the title

%----------------------------------------------------------------------------------------
%   CONTENT SECTION
%----------------------------------------------------------------------------------------

% - -- -  - -- -  - -- -  -
\section{Crafting A Compiler}
\subsection{Exercise 1.11 MOSS}
MOSS, measure of software similarity, interrupts code for its
   semantic meaning rather than its raw contents.
A neive approach to detect software similarity would be to
   to compare lines matching similar strings of text.
This is very similar to how turnitin and text plagiarism
   checkers works.
In the programming realm, avoiding similarity detection from that
   method would be simple: change variable names and code organization.
MOSS, however, works to interrupt the code and extract its meaning.
It achieves this with tokenization, creating an abstract syntax tree
   (AST), and semantic analysis.
MOSS then compares the resulting structures with other code bases
   that are also interpreted in this way.
Ultimately, this allows MOSS to find code which does the same thing
   and does it the same exact or similar way.
\newpage
\subsection{Exercise 3.1 Token Sequence}
\textless id, main\textgreater
\textless close\_paren\textgreater
\textless open\_block\textgreater\\

\textless type, float\textgreater
\textless id, bal\textgreater
\textless end\_statement\textgreater\\

\textless type, int\textgreater
\textless id, month\textgreater
\textless assignment\textgreater
\textless literal\_int, 0\textgreater
\textless end\_statement\textgreater\\

\textless id, bal\textgreater
\textless assignment\textgreater
\textless literal\_int, 15000\textgreater
\textless end\_statement\textgreater\\

\textless while\textgreater
\textless open\_paren\textgreater
\textless id, bal\textgreater
\textless greater\_than\textgreater
\textless literal\_int, 0\textgreater
\textless open\_paren\textgreater
\textless open\_block\textgreater\\

\textless id printf\textgreater
\textless open\_paren\textgreater
\textless literal\_str, Month: \%2d Balance: \%10.2f\\n\textgreater
\textless separator\textgreater
\textless id, month\textgreater
\textless separator\textgreater
\textless id, bal\textgreater
\textless close\_paren\textgreater\\

\textless id, month\textgreater
\textless assignment\textgreater
\textless id, month\textgreater
\textless add\_operator\textgreater
\textless literal\_int, 1\textgreater
\textless end\_statement\textgreater\\

\textless close\_block\textgreater
\textless close\_block\textgreater\\
The tokens that carry extra information are identifiers (id) and all literals.
Added information for observibility such as line and columns numbers for each
   token would also be important for a lexer to provide for future compiler steps.

\section{Dragon}
\subsection{Exercise 1.1.4 C Target Language}
The benefit of translating a language to C is that it is a well-established
   language with many existing compilers for a wide range of operating systems.
This means that only the frontend of a compiler would need to be built out because
   the existing C compilers can translate it to every supported OS.
\newpage
\subsection{Exercise 1.6.1 Variables in Block Code}
\begin{lstlisting}[language=C]
int w, x, y, z;
int i = 4; int j = 5;
{   int j = 7;
    i = 6;
    w = i + j;
}
x = i + j;
{   int i = 8;
    y = i + j;
}
z = i + j;
\end{lstlisting}
w is assigned i + j which is 6 + 7, so w = 13.\\
x is assigned i + j which is 6 + 5, so w = 11.\\
y is assigned i + j which is 8 + 5, so w = 13.\\
z is assigned i + j which is 6 + 5, so w = 11.\\
\end{document}

