%%%%%%%%%%%%%%%%%%%%%%%%%%%%%%%%%%%%%%%%%
%
% CMPT 435
% Lab Zero
%
%%%%%%%%%%%%%%%%%%%%%%%%%%%%%%%%%%%%%%%%%

%%%%%%%%%%%%%%%%%%%%%%%%%%%%%%%%%%%%%%%%%
% Short Sectioned Assignment
% LaTeX Template
% Version 1.0 (5/5/12)
%
% This template has been downloaded from: http://www.LaTeXTemplates.com
% Original author: % Frits Wenneker (http://www.howtotex.com)
% License: CC BY-NC-SA 3.0 (http://creativecommons.org/licenses/by-nc-sa/3.0/)
% Modified by Alan G. Labouseur  - alan@labouseur.com, and Patrick Tyler - Patrick.Tyler1@marist.edu
%
%%%%%%%%%%%%%%%%%%%%%%%%%%%%%%%%%%%%%%%%%

%----------------------------------------------------------------------------------------
%	PACKAGES AND OTHER DOCUMENT CONFIGURATIONS
%----------------------------------------------------------------------------------------

\documentclass[letterpaper, 10pt]{article} 

\usepackage[english]{babel} % English language/hyphenation
\usepackage{graphicx}
\usepackage[lined,linesnumbered,commentsnumbered]{algorithm2e}
\usepackage{listings}
\usepackage{fancyhdr} % Custom headers and footers
\pagestyle{fancyplain} % Makes all pages in the document conform to the custom headers and footers
\usepackage{lastpage}
\usepackage{url}
\usepackage{xcolor}
\usepackage{titlesec}
\usepackage{xurl}
\usepackage{multicol}
\usepackage{fancyvrb}
% Stolen from https://www.overleaf.com/learn/latex/Code_listing 
\definecolor{codegreen}{rgb}{0,0.6,0}
\definecolor{codegray}{rgb}{0.5,0.5,0.5}
\definecolor{codepurple}{rgb}{0.58,0,0.82}
\definecolor{backcolour}{rgb}{0.95,0.5,0.2}

\lstdefinestyle{mystyle}{
    backgroundcolor=\color{backcolour},   
    commentstyle=\color{codegreen},
    keywordstyle=\color{magenta},
    numberstyle=\tiny\color{codegray},
    stringstyle=\color{codepurple},
    basicstyle=\ttfamily\footnotesize,
    breakatwhitespace=false,         
    breaklines=true,                 
    captionpos=b,                    
    keepspaces=true,                 
    numbers=left,                    
    numbersep=5pt,                  
    showspaces=false,                
    showstringspaces=false,
    showtabs=false,                  
    tabsize=2
}
\lstset{style=mystyle, language=c++}


\fancyhead{} % No page header - if you want one, create it in the same way as the footers below
\fancyfoot[L]{} % Empty left footer
\fancyfoot[R]{}

\renewcommand{\headrulewidth}{0pt} % Remove header underlines
\renewcommand{\footrulewidth}{0pt} % Remove footer underlines
\setlength{\headheight}{13.6pt} % Customize the height of the header

%----------------------------------------------------------------------------------------
%	TITLE SECTION
%----------------------------------------------------------------------------------------

\newcommand{\horrule}[1]{\rule{\linewidth}{#1}} % Create horizontal rule command with 1 argument of height

\title{	
   \normalfont \normalsize 
   \textsc{CMPT 308 - Spring 2024 - Dr. Labouseur} \\[10pt] % Header stuff.
   \horrule{0.5pt} \\[0.25cm] 	% Top horizontal rule
   \huge Lab 3-5 -- ~Lexer \\     	    % Assignment title
   \horrule{0.5pt} \\[0.25cm] 	% Bottom horizontal rule
}

\author{Patrick Tyler \\ \normalsize Patrick.Tyler1@marist.edu}

\date{\normalsize\today} 	% Today's date.

\begin{document}

\maketitle % Print the title

%----------------------------------------------------------------------------------------
%   CONTENT SECTION
%----------------------------------------------------------------------------------------

% - -- -  - -- -  - -- -  -
\section{Crafting A Compiler}
\subsection{4.7 derivations}
\begin{verbatim}
A)
E $ 
T plus E $ 
F plus E $ 
num plus E $ 
num plus T plus E $ 
num plus T times F plus E $ 
num plus F times F plus E $ 
num plus num times num plus E $ 
num plus num times num plus E $ 
num plus num times num plus T $ 
num plus num times num plus F $ 
num plus num times num plus num $ 
B)
E $ 
T plus E $ 
T plus T plus E $ 
T plus T plus T $ 
T plus T plus F $ 
T plus T plus num $ 
T plus T times F plus num $ 
T plus T times num plus num $ 
T plus F times num plus num $ 
T plus num times num plus num $ 
F plus num times num plus num $ 
num plus num times num plus num $ 
\end{verbatim}
c)\\
The left to right associatvity of operators can derive the same
set of terminal symbols with different trees. If I actually
followed the rules it probably would have happend this way.
This is because the langauge is not LL1.
\subsection{5.2c recursive parser}
\begin{verbatim}
let tokens = new Tokens;
let current_node = new Node;
let tree = new Tree(current_node);

fn match(token) {
    if tokens.next() == token {
        tree.add(token);
    } else {
        err;
    }
}

fn move_up() {
    current_node = current_node.parent;
}

fn do_start() {
    do_value();
    match($);
    move_up();
}

fn do_value() {
    next_token = tokens.peek();
    if (next_token == num) {
        match(num);
    } else if (next_token == |paren) {
        match(lparen);
        do_expr();
        match(rparen);
    } else {
        err;
    }
    move_up():
}

fn do_expr() {
    next_token = tokens.peek();
    if (next_token == plus) {
        match(plus);
        do_value()
        do_value()
    } else if (next_token == prod) {
        match(prod);
        do_values()
    } else {
        err;
    }
    move_up():
}

fn do_values() {
    next_token = tokens.peek();
    if (next_token in [num, lparen]) {
        do_value();
        do_values();
    }
    move_up():
}
\end{verbatim}
\section{Dragon}
\subsection{Exercise 4.2.1}
\begin{verbatim}
A)
S
SS*
SS+S*
aS+S*
aa+S*
aa+a*
B)
S
Sa*
SS+a*
Sa+a*
aa+a*
\end{verbatim}
\newpage
\begin{verbatim}
C)
S
- S
-- S
--- a
-- S
--- a
-- +
- S
-- a
- *
\end{verbatim}
\end{document}

